\chapter{Manual Técnico y Despliegue}

\section{Estructura del Proyecto}
El código fuente de la aplicación se organiza siguiendo el patrón de diseño de Flask \cite{flask}. A continuación se muestra la estructura de directorios principal y la función de cada módulo:

\begin{verbatim}
/tfg-web-ci-python
|-- .github/            # Flujos de CI/CD (GitHub Actions)
|-- app/
|   |-- data/           # Datos estáticos (JSON)
|   |-- static/         # CSS, JS e imágenes
|   |-- templates/      # Plantillas HTML (Jinja2)
|   |-- __init__.py     # Factoría de la aplicación
|   |-- career.py       # Lógica del "Modo Carrera"
|   `-- routes.py       # Controlador (Rutas)
|-- tests/              # Tests unitarios (pytest)
|-- requirements.txt    # Dependencias
`-- run.py              # Entry point
\end{verbatim}

\section{Instalación Local}
Para ejecutar el entorno de desarrollo se requieren Python 3.11+ \cite{python} y Git \cite{progit}.

\begin{enumerate}
    \item Clonar el repositorio:
    \begin{verbatim}
    git clone https://github.com/sanlaja/tfg-web-ci-python.git
    \end{verbatim}
    
    \item Crear entorno virtual e instalar dependencias:
    \begin{verbatim}
    python -m venv .venv
    source .venv/bin/activate  # Windows: .venv\Scripts\activate
    pip install -r requirements.txt
    \end{verbatim}
    
    \item Ejecutar la aplicación:
    \begin{verbatim}
    python run.py
    \end{verbatim}
\end{enumerate}

Si la ejecución es correcta, la aplicación estará disponible localmente en:
\begin{center}
\texttt{http://localhost:5000}
\end{center}

Para acceder a la versión de producción desplegada en la nube (Render \cite{render}), se puede visitar:
\begin{center}
\texttt{https://tfg-web-ci-python.onrender.com}
\end{center}