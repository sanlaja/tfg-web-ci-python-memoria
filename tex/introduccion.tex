% Contenidos del capítulo.
% Las secciones presentadas son orientativas y no representan
% necesariamente la organización que debe tener este capítulo.

\section{Introducción}

La inversión en mercados financieros ha ganado popularidad en los últimos años, facilitada por el acceso a plataformas digitales. Sin embargo, la falta de educación financiera lleva a muchos inversores novatos a cometer errores costosos. Según estudios clásicos de Barber y Odean, basados en más de 60.000 hogares estadounidenses, los inversores particulares que operan con mayor frecuencia obtienen rentabilidades anuales entre 5 y 10 puntos porcentuales inferiores a las de un índice de mercado amplio, principalmente debido a sesgos conductuales como el exceso de confianza y a una operativa excesiva \cite{barber2000trading}.

Este TFG presenta el diseño e implementación de una aplicación web orientada al análisis fundamental de empresas y la simulación de estrategias de inversión a largo plazo. La herramienta, denominada "TFG Web App", permite a los usuarios practicar estrategias como el \textit{Dollar Cost Averaging} (DCA) en un entorno libre de riesgo, utilizando datos históricos reales del mercado obtenidos a través de Yahoo Finance \cite{yfinance}.

A diferencia de los simuladores de \textit{trading} tradicionales, que suelen fomentar la especulación a corto plazo, esta propuesta se centra en la inversión sosegada y el análisis de métricas fundamentales (PER, deuda, crecimiento), ofreciendo un "Modo Carrera" gamificado para evaluar el desempeño del usuario ante distintos escenarios macroeconómicos.

\section{Estructura de la memoria}

El resto de este documento se estructura de la siguiente manera:

\begin{itemize}
    \item En el \textbf{Capítulo 2} se detallan la motivación del proyecto y los objetivos principales y específicos que se pretenden alcanzar.
    \item El \textbf{Capítulo 3} revisa el estado del arte, analizando las herramientas existentes y justificando las tecnologías seleccionadas (Python, Flask, CI/CD).
    \item El \textbf{Capítulo 4} aborda la planificación del proyecto, incluyendo la especificación de requisitos y la estimación de costes.
    \item El \textbf{Capítulo 5} describe el desarrollo del sistema, desde el análisis y diseño de la arquitectura hasta los detalles de implementación.
    \item El \textbf{Capítulo 6} presenta las pruebas realizadas y los resultados obtenidos.
    \item Finalmente, el \textbf{Capítulo 7} expone las conclusiones y las líneas de trabajo futuro.
\end{itemize}